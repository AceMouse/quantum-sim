\section{Motivation}
The Fourier transfrom is one of the most widely applicable procedures, used in anything from signal processing and AI to seismology. There for it is of interest to explore different ways of computing it. 
Our most efficient algorithm for computing this classicaly has been the fast Fourier transform (FFT). There is an analogus algorithm designed for quantum computers namely the quantum Fourier transform (QFT). 
This is also widely used as a component of other quantum algorithms such as Shor's algorithm, and is therefor an intersting subject of improvement. 
The QFT has a significantly better time complexity, but only when run on a quantum computer. The QFT takes $O(n^2)$ quantum gates and FFT takes $O(2^n log_{2}(2^n))$ classical logic gates to compute for $2^n$ amplitudes or data points, for the QFT $n$ corresponds to the required number of qubits. 
Recently there has been advancements in simulating the QFT approximatly on classical computers, while retaining some of this speedup, which is what this project focuses on. We will go though the math needed to implement two different simulations of the QFT, one exact and one approximate but vastly faster and more memmory efficient. The first only uses basic matrix math, the second requires a basic understanding of tensor networks which we will try to provide. This project also includes code implementing all the math touched upon by this paper and is intended as a learning resource, this means that the formulas shown will often not be the pretty driviations seen elsewhere but will seek to repressent the individual bits that went into constructing them.  
