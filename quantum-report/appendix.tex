\section{Appendix}
\subsection{calculations for section \ref{sec:maniq}} \label{app:maniq}
\begin{figure}[H]
    \centering
    \begin{align*}
        \qti\ket{\qtz}&=\qi\qz=
        \begin{bmatrix}
            1 \cdot 1 + 0 \cdot 0\\
            0 \cdot 1 + 1 \cdot 0\\
        \end{bmatrix}=
        \qz=\ket{\qtz}\\
        \qti\ket{\qto}&=\qi\qo=
        \begin{bmatrix}
            1 \cdot 0 + 0 \cdot 1\\
            0 \cdot 0 + 1 \cdot 1\\
        \end{bmatrix}=
        \qo=\ket{\qto}
    \end{align*}
    \caption{Applying $\qti$ gate to the states $\ket{\qtz}$ and $\ket{\qto}$}
    \label{fig:app_i}
\end{figure}

\begin{figure}[H]
    \centering
    \begin{align*}
        \qtx\ket{\qtz}&=\qx\qz=
        \begin{bmatrix}
            0 \cdot 1 + 1 \cdot 0\\
            1 \cdot 1 + 0 \cdot 0\\
        \end{bmatrix}=
        \qo=\ket{\qto}\\
        \qtx\ket{\qto}&=\qx\qo=
        \begin{bmatrix}
            0 \cdot 0 + 1 \cdot 1\\
            1 \cdot 0 + 0 \cdot 1\\
        \end{bmatrix}=
        \qz=\ket{\qtz}
    \end{align*}
    \caption{Applying $\qtx$ gate to the states $\ket{\qtz}$ and $\ket{\qto}$}
    \label{fig:app_x}
\end{figure}

\begin{figure}[H]
    \centering
    \begin{align*}
        \qth\ket{\qtz}&=\qh\qz=
        \begin{bmatrix}
            \sqrt{0.5} \cdot 1 + \sqrt{0.5} \cdot 0\\
            \sqrt{0.5} \cdot 1 - \sqrt{0.5} \cdot 0\\
        \end{bmatrix}=
        \qp=\ket{\qtp}\\
        \qth\ket{\qto}&=\qh\qo=
        \begin{bmatrix}
            \sqrt{0.5} \cdot 0 + \sqrt{0.5} \cdot 1\\
            \sqrt{0.5} \cdot 0 - \sqrt{0.5} \cdot 1\\
        \end{bmatrix}=
        \qm=\ket{\qtm}
    \end{align*}
    \caption{Applying $\qth$ gate to the states $\ket{\qtz}$ and $\ket{\qto}$}
    \label{fig:app_h}
\end{figure}

\begin{figure}[H]
    \centering
    \begin{align*}
        \qtr{2}\ket{\qtz}&=\qr{2}\qz=
        \begin{bmatrix}
            1 \cdot 1 + 0 \cdot 0\\
            0 \cdot 1 + \qre{2} \cdot 0\\
        \end{bmatrix}=
        \qz=\ket{\qtz}\\
        \qtr{2}\ket{\qto}&=\qr{2}\qo=
        \begin{bmatrix}
            1 \cdot 0 + 0 \cdot 1\\
            0 \cdot 0 + \qre{2} \cdot 1\\
        \end{bmatrix}=
        \begin{bmatrix}
            0\\
            \qre{2}\\
        \end{bmatrix}=
        \begin{bmatrix}
            0\\
            \mathrm{i}\\
        \end{bmatrix}
    \end{align*}
    \caption{Applying $\qtr{2}$ gate to the states $\ket{\qtz}$ and $\ket{\qto}$}
    \label{fig:app_r2}
\end{figure}

\begin{figure}[H]
    \centering
    \begin{align*}
        \qtcx\ket{\qtz\qtz}&=\qcx
        \begin{bmatrix}
            1\\
            0\\
            0\\
            0\\
        \end{bmatrix}
        =
        \begin{bmatrix}
            1\cdot 1 + 0\cdot 0 + 0\cdot 0 + 0\cdot 0\\
            0\cdot 1 + 1\cdot 0 + 0\cdot 0 + 0\cdot 0\\
            0\cdot 1 + 0\cdot 0 + 0\cdot 0 + 1\cdot 0\\
            0\cdot 1 + 0\cdot 0 + 1\cdot 0 + 0\cdot 0\\
        \end{bmatrix}=
        \begin{bmatrix}
            1\\
            0\\
            0\\
            0\\
        \end{bmatrix}
        =\ket{\qtz\qtz}\\
        \qtcx\ket{\qtz\qto}&=\qcx
        \begin{bmatrix}
            0\\
            1\\
            0\\
            0\\
        \end{bmatrix}
        =
        \begin{bmatrix}
            1\cdot 0 + 0\cdot 1 + 0\cdot 0 + 0\cdot 0\\
            0\cdot 0 + 1\cdot 1 + 0\cdot 0 + 0\cdot 0\\
            0\cdot 0 + 0\cdot 1 + 0\cdot 0 + 1\cdot 0\\
            0\cdot 0 + 0\cdot 1 + 1\cdot 0 + 0\cdot 0\\
        \end{bmatrix}=
        \begin{bmatrix}
            0\\
            1\\
            0\\
            0\\
        \end{bmatrix}
        =\ket{\qtz\qto}\\
        \qtcx\ket{\qto\qtz}&=\qcx
        \begin{bmatrix}
            0\\
            0\\
            1\\
            0\\
        \end{bmatrix}
        =
        \begin{bmatrix}
            1\cdot 0 + 0\cdot 0 + 0\cdot 1 + 0\cdot 0\\
            0\cdot 0 + 1\cdot 0 + 0\cdot 1 + 0\cdot 0\\
            0\cdot 0 + 0\cdot 0 + 0\cdot 1 + 1\cdot 0\\
            0\cdot 0 + 0\cdot 0 + 1\cdot 1 + 0\cdot 0\\
        \end{bmatrix}=
        \begin{bmatrix}
            0\\
            0\\
            0\\
            1\\
        \end{bmatrix}
        =\ket{\qto\qto}\\
        \qtcx\ket{\qto\qto}&=\qcx
        \begin{bmatrix}
            0\\
            0\\
            0\\
            1\\
        \end{bmatrix}
        =
        \begin{bmatrix}
            1\cdot 0 + 0\cdot 0 + 0\cdot 0 + 0\cdot 1\\
            0\cdot 0 + 1\cdot 0 + 0\cdot 0 + 0\cdot 1\\
            0\cdot 0 + 0\cdot 0 + 0\cdot 0 + 1\cdot 1\\
            0\cdot 0 + 0\cdot 0 + 1\cdot 0 + 0\cdot 1\\
        \end{bmatrix}=
        \begin{bmatrix}
            0\\
            0\\
            1\\
            0\\
        \end{bmatrix}
        =\ket{\qto\qtz}
    \end{align*}
    \caption{Applying $\qtcx$ gate to the states $\ket{\qtz\qtz}$, $\ket{\qtz\qto}$, $\ket{\qto\qtz}$ and $\ket{\qto\qto}$}
    \label{fig:app_cx}
\end{figure}

\subsection{calculations for section \ref{sec:maniqi}}\label{app:maniqi}
\begin{figure}[H]
    \centering
    \begin{align*}
        \left(\qti\otimes\qti\otimes\qtx\right)\ket{\qtz\qtz\qtz}=\left(\qi\otimes\qi\otimes\qx\right)\left(\qz\otimes\qz\otimes\qz\right)&= \\
        \begin{bmatrix}
            1
            \begin{bmatrix}
                1 \qx & 0 \qx\\
                0 \qx & 1 \qx\\
            \end{bmatrix}
            & 0
            \begin{bmatrix}
                1 \qx & 0 \qx\\
                0 \qx & 1 \qx\\
            \end{bmatrix}\\
            0 
            \begin{bmatrix}
                1 \qx & 0 \qx\\
                0 \qx & 1 \qx\\
            \end{bmatrix}
            & 1
            \begin{bmatrix}
                1 \qx & 0 \qx\\
                0 \qx & 1 \qx\\
            \end{bmatrix}\\ 
        \end{bmatrix}
        \begin{bmatrix}
            1\\
            0\\
            0\\
            0\\
            0\\
            0\\
            0\\
            0
        \end{bmatrix}
        &=\\
        \begin{bmatrix}
            0 & 1 & 0 & 0 & 0 & 0 & 0 & 0\\
            1 & 0 & 0 & 0 & 0 & 0 & 0 & 0\\
            0 & 0 & 0 & 1 & 0 & 0 & 0 & 0\\
            0 & 0 & 1 & 0 & 0 & 0 & 0 & 0\\
            0 & 0 & 0 & 0 & 0 & 1 & 0 & 0\\
            0 & 0 & 0 & 0 & 1 & 0 & 0 & 0\\
            0 & 0 & 0 & 0 & 0 & 0 & 0 & 1\\
            0 & 0 & 0 & 0 & 0 & 0 & 1 & 0\\
        \end{bmatrix}
        \begin{bmatrix}
            1\\
            0\\
            0\\
            0\\
            0\\
            0\\
            0\\
            0
        \end{bmatrix}
        &=\\
        \begin{bmatrix}
            0\cdot 1 + 1\cdot 0 + 0\cdot 0 + 0\cdot 0 + 0\cdot 0 + 0\cdot 0 + 0\cdot 0 + 0\cdot 0\\
            1\cdot 1 + 0\cdot 0 + 0\cdot 0 + 0\cdot 0 + 0\cdot 0 + 0\cdot 0 + 0\cdot 0 + 0\cdot 0\\
            0\cdot 1 + 0\cdot 0 + 0\cdot 0 + 1\cdot 0 + 0\cdot 0 + 0\cdot 0 + 0\cdot 0 + 0\cdot 0\\
            0\cdot 1 + 0\cdot 0 + 1\cdot 0 + 0\cdot 0 + 0\cdot 0 + 0\cdot 0 + 0\cdot 0 + 0\cdot 0\\
            0\cdot 1 + 0\cdot 0 + 0\cdot 0 + 0\cdot 0 + 0\cdot 0 + 1\cdot 0 + 0\cdot 0 + 0\cdot 0\\
            0\cdot 1 + 0\cdot 0 + 0\cdot 0 + 0\cdot 0 + 1\cdot 0 + 0\cdot 0 + 0\cdot 0 + 0\cdot 0\\
            0\cdot 1 + 0\cdot 0 + 0\cdot 0 + 0\cdot 0 + 0\cdot 0 + 0\cdot 0 + 0\cdot 0 + 1\cdot 0\\
            0\cdot 1 + 0\cdot 0 + 0\cdot 0 + 0\cdot 0 + 0\cdot 0 + 0\cdot 0 + 1\cdot 0 + 0\cdot 0\\
        \end{bmatrix}
        =
        \begin{bmatrix}
            0\\
            1\\
            0\\
            0\\
            0\\
            0\\
            0\\
            0
        \end{bmatrix}
        &=
        \ket{\qtz\qtz\qto}\\
    \end{align*}
    \caption{Applying $\qtx$ to the first qubit in a 3 qubit system}
    \label{fig:app_individual_application1}
\end{figure}


\begin{figure}[H]
    \centering
    \begin{align*}
        \left(\qti\otimes\qtx\otimes\qti\right)\ket{\qtz\qtz\qtz}=\left(\qi\otimes\qx\otimes\qi\right)\left(\qz\otimes\qz\otimes\qz\right)&= \\
        \begin{bmatrix}
            1
            \begin{bmatrix}
                0 \qi & 1 \qi\\
                1 \qi & 0 \qi\\
            \end{bmatrix}
            & 0
            \begin{bmatrix}
                0 \qi & 1 \qi\\
                1 \qi & 0 \qi\\
            \end{bmatrix}\\
            0 
            \begin{bmatrix}
                0 \qi & 1 \qi\\
                1 \qi & 0 \qi\\
            \end{bmatrix}
            & 1
            \begin{bmatrix}
                0 \qi & 1 \qi\\
                1 \qi & 0 \qi\\
            \end{bmatrix}\\ 
        \end{bmatrix}
        \begin{bmatrix}
            1\\
            0\\
            0\\
            0\\
            0\\
            0\\
            0\\
            0
        \end{bmatrix}
        &=\\
        \begin{bmatrix}
            0 & 0 & 1 & 0 & 0 & 0 & 0 & 0\\
            0 & 0 & 0 & 1 & 0 & 0 & 0 & 0\\
            1 & 0 & 0 & 0 & 0 & 0 & 0 & 0\\
            0 & 1 & 0 & 0 & 0 & 0 & 0 & 0\\
            0 & 0 & 0 & 0 & 0 & 0 & 1 & 0\\
            0 & 0 & 0 & 0 & 0 & 0 & 0 & 1\\
            0 & 0 & 0 & 0 & 1 & 0 & 0 & 0\\
            0 & 0 & 0 & 0 & 0 & 1 & 0 & 0\\
        \end{bmatrix}
        \begin{bmatrix}
            1\\
            0\\
            0\\
            0\\
            0\\
            0\\
            0\\
            0
        \end{bmatrix}
        &=\\
        \begin{bmatrix}
            0\cdot 1 + 0\cdot 0 + 1\cdot 0 + 0\cdot 0 + 0\cdot 0 + 0\cdot 0 + 0\cdot 0 + 0\cdot 0\\
            0\cdot 1 + 0\cdot 0 + 0\cdot 0 + 1\cdot 0 + 0\cdot 0 + 0\cdot 0 + 0\cdot 0 + 0\cdot 0\\
            1\cdot 1 + 0\cdot 0 + 0\cdot 0 + 0\cdot 0 + 0\cdot 0 + 0\cdot 0 + 0\cdot 0 + 0\cdot 0\\
            0\cdot 1 + 1\cdot 0 + 0\cdot 0 + 0\cdot 0 + 0\cdot 0 + 0\cdot 0 + 0\cdot 0 + 0\cdot 0\\
            0\cdot 1 + 0\cdot 0 + 0\cdot 0 + 0\cdot 0 + 0\cdot 0 + 0\cdot 0 + 1\cdot 0 + 0\cdot 0\\
            0\cdot 1 + 0\cdot 0 + 0\cdot 0 + 0\cdot 0 + 0\cdot 0 + 0\cdot 0 + 0\cdot 0 + 1\cdot 0\\
            0\cdot 1 + 0\cdot 0 + 0\cdot 0 + 0\cdot 0 + 1\cdot 0 + 0\cdot 0 + 0\cdot 0 + 0\cdot 0\\
            0\cdot 1 + 0\cdot 0 + 0\cdot 0 + 0\cdot 0 + 0\cdot 0 + 1\cdot 0 + 0\cdot 0 + 0\cdot 0\\
        \end{bmatrix}
        =
        \begin{bmatrix}
            0\\
            0\\
            1\\
            0\\
            0\\
            0\\
            0\\
            0
        \end{bmatrix}
        &=
        \ket{\qtz\qto\qtz}\\
    \end{align*}
    \caption{Applying $\qtx$ to the second qubit in a 3 qubit system}
    \label{fig:app_individual_application2}
\end{figure}
\begin{figure}[H]
    \centering
    \begin{align*}
        \left(\qtx\otimes\qti\otimes\qti\right)\ket{\qtz\qtz\qtz}=\left(\qx\otimes\qi\otimes\qi\right)\left(\qz\otimes\qz\otimes\qz\right)&= \\
        \begin{bmatrix}
            0
            \begin{bmatrix}
                1 \qi & 0 \qi\\
                0 \qi & 1 \qi\\
            \end{bmatrix}
            & 1
            \begin{bmatrix}
                1 \qi & 0 \qi\\
                0 \qi & 1 \qi\\
           \end{bmatrix}\\
           1 
           \begin{bmatrix}
               1 \qi & 0 \qi\\
                0 \qi & 1 \qi\\
            \end{bmatrix}
            & 0
            \begin{bmatrix}
                1 \qi & 0 \qi\\
                0 \qi & 1 \qi
            \end{bmatrix}\\ 
        \end{bmatrix}
        \begin{bmatrix}
            1\\
            0\\
            0\\
            0\\
            0\\
            0\\
            0\\
            0
        \end{bmatrix}
        &=\\
        \begin{bmatrix}
            0 & 0 & 0 & 0 & 1 & 0 & 0 & 0\\
            0 & 0 & 0 & 0 & 0 & 1 & 0 & 0\\
            0 & 0 & 0 & 0 & 0 & 0 & 1 & 0\\
            0 & 0 & 0 & 0 & 0 & 0 & 0 & 1\\
            1 & 0 & 0 & 0 & 0 & 0 & 0 & 0\\
            0 & 1 & 0 & 0 & 0 & 0 & 0 & 0\\
            0 & 0 & 1 & 0 & 0 & 0 & 0 & 0\\
            0 & 0 & 0 & 1 & 0 & 0 & 0 & 0\\
        \end{bmatrix}
        \begin{bmatrix}
            1\\
            0\\
            0\\
            0\\
            0\\
            0\\
            0\\
            0
        \end{bmatrix}
        &=\\
        \begin{bmatrix}
            0\cdot 1 + 0\cdot 0 + 0\cdot 0 + 0\cdot 0 + 1\cdot 0 + 0\cdot 0 + 0\cdot 0 + 0\cdot 0\\
            0\cdot 1 + 0\cdot 0 + 0\cdot 0 + 0\cdot 0 + 0\cdot 0 + 1\cdot 0 + 0\cdot 0 + 0\cdot 0\\
            0\cdot 1 + 0\cdot 0 + 0\cdot 0 + 0\cdot 0 + 0\cdot 0 + 0\cdot 0 + 1\cdot 0 + 0\cdot 0\\
            0\cdot 1 + 0\cdot 0 + 0\cdot 0 + 0\cdot 0 + 0\cdot 0 + 0\cdot 0 + 0\cdot 0 + 1\cdot 0\\
            1\cdot 1 + 0\cdot 0 + 0\cdot 0 + 0\cdot 0 + 0\cdot 0 + 0\cdot 0 + 0\cdot 0 + 0\cdot 0\\
            0\cdot 1 + 1\cdot 0 + 0\cdot 0 + 0\cdot 0 + 0\cdot 0 + 0\cdot 0 + 0\cdot 0 + 0\cdot 0\\
            0\cdot 1 + 0\cdot 0 + 1\cdot 0 + 0\cdot 0 + 0\cdot 0 + 0\cdot 0 + 0\cdot 0 + 0\cdot 0\\
            0\cdot 1 + 0\cdot 0 + 0\cdot 0 + 1\cdot 0 + 0\cdot 0 + 0\cdot 0 + 0\cdot 0 + 0\cdot 0\\
        \end{bmatrix}
        =
        \begin{bmatrix}
            0\\
            0\\
            0\\
            0\\
            1\\
            0\\
            0\\
            0
        \end{bmatrix}
        &=
        \ket{\qto\qtz\qtz}
    \end{align*}
    \caption{Applying $\qtx$ to the thrid qubit in a 3 qubit system}
    \label{fig:app_individual_application3}
\end{figure}
\newpage
\noindent
If we want to apply multiple gates at a time we just replace the corresponding $\qti$ gates. 
\begin{figure}[H]
    \centering
    \begin{align*}
        \left(\qtx\otimes\qti\otimes\qth\right)\ket{\qtz\qtz\qtz}=\left(\qx\otimes\qi\otimes\qh\right)\left(\qz\otimes\qz\otimes\qz\right)&= \\
        \begin{bmatrix}
            0
            \begin{bmatrix}
                1 \qh & 0 \qh\\
                0 \qh & 1 \qh\\
            \end{bmatrix}
            & 1
            \begin{bmatrix}
                1 \qh & 0 \qh\\
                0 \qh & 1 \qh\\
            \end{bmatrix}\\
            1 
            \begin{bmatrix}
                1 \qh & 0 \qh\\
                0 \qh & 1 \qh\\
            \end{bmatrix}
            & 0
            \begin{bmatrix}
                1 \qh & 0 \qh\\
                0 \qh & 1 \qh\\
            \end{bmatrix}\\ 
        \end{bmatrix}
        \begin{bmatrix}
            1\\
            0\\
            0\\
            0\\
            0\\
            0\\
            0\\
            0
        \end{bmatrix}
        &=\\
        \begin{bmatrix}
        0           &  0          &   0          &   0          &   \sqrt{0.5}      &          \sqrt{0.5} &   0          &   0         \\
        0           &  0          &   0          &   0          &   \sqrt{0.5}      &         -\sqrt{0.5} &   0          &   0         \\
        0           &  0          &   0          &   0          &   0               &          0          &   \sqrt{0.5} &   \sqrt{0.5}\\
        0           &  0          &   0          &   0          &   0               &          0          &   \sqrt{0.5} &  -\sqrt{0.5}\\
        \sqrt{0.5}  &  \sqrt{0.5} &   0          &   0          &   0               &          0          &   0          &   0         \\
        \sqrt{0.5}  & -\sqrt{0.5} &   0          &   0          &   0               &          0          &   0          &   0         \\
        0           &  0          &   \sqrt{0.5} &   \sqrt{0.5} &   0               &          0          &   0          &   0         \\
        0           &  0          &   \sqrt{0.5} &  -\sqrt{0.5} &   0               &          0          &   0          &   0         \\
        \end{bmatrix}
        \begin{bmatrix}
            1\\
            0\\
            0\\
            0\\
            0\\
            0\\
            0\\
            0
        \end{bmatrix}
        &=\\
        \begin{bmatrix}
            0           \cdot 1 +  0          \cdot 0 +   0          \cdot 0 +   0          \cdot 0 +   \sqrt{0.5}      \cdot 0 +          \sqrt{0.5} \cdot 0 +   0          \cdot 0 +   0         \cdot 0\\
            0           \cdot 1 +  0          \cdot 0 +   0          \cdot 0 +   0          \cdot 0 +   \sqrt{0.5}      \cdot 0           -\sqrt{0.5} \cdot 0 +   0          \cdot 0 +   0         \cdot 0\\
            0           \cdot 1 +  0          \cdot 0 +   0          \cdot 0 +   0          \cdot 0 +   0               \cdot 0 +          0          \cdot 0 +   \sqrt{0.5} \cdot 0 +   \sqrt{0.5}\cdot 0\\
            0           \cdot 1 +  0          \cdot 0 +   0          \cdot 0 +   0          \cdot 0 +   0               \cdot 0 +          0          \cdot 0 +   \sqrt{0.5} \cdot 0    -\sqrt{0.5}\cdot 0\\
            \sqrt{0.5}  \cdot 1 +  \sqrt{0.5} \cdot 0 +   0          \cdot 0 +   0          \cdot 0 +   0               \cdot 0 +          0          \cdot 0 +   0          \cdot 0 +   0         \cdot 0\\
            \sqrt{0.5}  \cdot 1   -\sqrt{0.5} \cdot 0 +   0          \cdot 0 +   0          \cdot 0 +   0               \cdot 0 +          0          \cdot 0 +   0          \cdot 0 +   0         \cdot 0\\
            0           \cdot 1 +  0          \cdot 0 +   \sqrt{0.5} \cdot 0 +   \sqrt{0.5} \cdot 0 +   0               \cdot 0 +          0          \cdot 0 +   0          \cdot 0 +   0         \cdot 0\\
            0           \cdot 1 +  0          \cdot 0 +   \sqrt{0.5} \cdot 0    -\sqrt{0.5} \cdot 0 +   0               \cdot 0 +          0          \cdot 0 +   0          \cdot 0 +   0         \cdot 0\\
        \end{bmatrix}
        =
        \begin{bmatrix}
            0\\
            0\\
            0\\
            0\\
            \sqrt{0.5}\\
            \sqrt{0.5}\\
            0\\
            0
        \end{bmatrix}
        &=
        \ket{\qto\qtz\qtp}
    \end{align*}
    \caption{Applying $\qtx$ to the thrid qubit and $\qth$ to the first qubit in a 3 qubit system}
    \label{fig:app_applying_multiple_gates}
\end{figure}

\begin{figure}[H]
    \centering
    \begin{align*}
        \mathbf{U}=\qtcx  \left(\qti\otimes\qth\right) \left(\qti\otimes\qtx\right)&=\\
        \qcx  
        \begin{bmatrix}
            1\qh & 0\qh \\
            0\qh & 1\qh \\
        \end{bmatrix}
        \begin{bmatrix}
            1\qx & 0\qx \\
            0\qx & 1\qx \\
        \end{bmatrix}
        &=\\
        \qcx  
        \begin{bmatrix}
            \sqrt{0.5} &  \sqrt{0.5} & 0 & 0 \\
            \sqrt{0.5} & -\sqrt{0.5} & 0 & 0 \\
            0 & 0 & \sqrt{0.5} &  \sqrt{0.5} \\
            0 & 0 & \sqrt{0.5} & -\sqrt{0.5} \\
        \end{bmatrix}
        \begin{bmatrix}
            0 & 1 & 0 & 0 \\
            1 & 0 & 0 & 0 \\
            0 & 0 & 0 & 1 \\
            0 & 0 & 1 & 0 \\
        \end{bmatrix}
        &=\\
        \begin{bmatrix}
            \sqrt{0.5} &  \sqrt{0.5} & 0 & 0 \\
            \sqrt{0.5} & -\sqrt{0.5} & 0 & 0 \\
            0 & 0 & \sqrt{0.5} & -\sqrt{0.5} \\
            0 & 0 & \sqrt{0.5} &  \sqrt{0.5} \\
        \end{bmatrix}
        \begin{bmatrix}
            0 & 1 & 0 & 0 \\
            1 & 0 & 0 & 0 \\
            0 & 0 & 0 & 1 \\
            0 & 0 & 1 & 0 \\
        \end{bmatrix}
        &=\\
        \begin{bmatrix}
            \sqrt{0.5}  &  \sqrt{0.5} & 0 & 0 \\
            -\sqrt{0.5} &  \sqrt{0.5} & 0 & 0 \\
            0 & 0 & -\sqrt{0.5} & \sqrt{0.5} \\
            0 & 0 & \sqrt{0.5}  & \sqrt{0.5} \\
        \end{bmatrix}& 
    \end{align*}
    \caption{Computing a matrix representation of a circuit,$\mathbf{U}$, that first applies $\qtx$ then $\qth$ to the first qubit then $\qtcx$ to the first and second qubit}
    \label{fig:app_circuit_matrix}
\end{figure}
\begin{figure}[H]
    \begin{alignat*}{2}
        \mathbf{U}\ket{\qtz\qtz}&=
        \begin{bmatrix}
            \sqrt{0.5}  &  \sqrt{0.5}   & 0             & 0          \\
            -\sqrt{0.5} &  \sqrt{0.5}   & 0             & 0          \\
            0           & 0             & -\sqrt{0.5}   & \sqrt{0.5} \\
            0           & 0             & \sqrt{0.5}    & \sqrt{0.5} \\
        \end{bmatrix}
        \begin{bmatrix}
            1 \\
            0 \\
            0 \\
            0 \\
        \end{bmatrix}
        =
        \begin{bmatrix}
            \sqrt{0.5}   \\
            -\sqrt{0.5}  \\
            0            \\
            0            \\
        \end{bmatrix}&=\ket{\qtz\qtm}\\
        \mathbf{U}\ket{\qtz\qto}&=
        \begin{bmatrix}
            \sqrt{0.5}  &  \sqrt{0.5}   & 0             & 0          \\
            -\sqrt{0.5} &  \sqrt{0.5}   & 0             & 0          \\
            0           & 0             & -\sqrt{0.5}   & \sqrt{0.5} \\
            0           & 0             & \sqrt{0.5}    & \sqrt{0.5} \\
        \end{bmatrix}
        \begin{bmatrix}
            0 \\
            1 \\
            0 \\
            0 \\
        \end{bmatrix}
        =
        \begin{bmatrix}
            \sqrt{0.5}   \\
            \sqrt{0.5}   \\
            0            \\
            0            \\
        \end{bmatrix}&=\ket{\qtz\qtp}\\
        \mathbf{U}\ket{\qto\qtz}&=
        \begin{bmatrix}
            \sqrt{0.5}  &  \sqrt{0.5}   & 0             & 0          \\
            -\sqrt{0.5} &  \sqrt{0.5}   & 0             & 0          \\
            0           & 0             & -\sqrt{0.5}   & \sqrt{0.5} \\
            0           & 0             & \sqrt{0.5}    & \sqrt{0.5} \\
        \end{bmatrix}
        \begin{bmatrix}
            0 \\
            0 \\
            1 \\
            0 \\
        \end{bmatrix}
        =
        \begin{bmatrix}
            0            \\
            0            \\
            -\sqrt{0.5}  \\
            \sqrt{0.5}   \\
        \end{bmatrix}&=-\ket{\qto\qtm} \\
        \mathbf{U}\ket{\qto\qto}&=
        \begin{bmatrix}
            \sqrt{0.5}  &  \sqrt{0.5}   & 0             & 0          \\
            -\sqrt{0.5} &  \sqrt{0.5}   & 0             & 0          \\
            0           & 0             & -\sqrt{0.5}   & \sqrt{0.5} \\
            0           & 0             & \sqrt{0.5}    & \sqrt{0.5} \\
        \end{bmatrix}
        \begin{bmatrix}
            0 \\
            0 \\
            0 \\
            1 \\
        \end{bmatrix}
        =
        \begin{bmatrix}
            0             \\
            0             \\
            \sqrt{0.5}    \\
            \sqrt{0.5}    \\
        \end{bmatrix}&=\ket{\qto\qtp}
    \end{alignat*}
    \caption{Applying the circuit represented by the matrix $\mathbf{U}$ from figure \ref{fig:app_circuit_matrix} to different states}
    \label{fig:app_applying_circuit}
\end{figure}
